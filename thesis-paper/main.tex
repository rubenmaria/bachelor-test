\documentclass[12pt,letterpaper]{article}
%\usepackage[letterpaper,top=3.0cm, bottom=3.0cm, footnotesep=1.0cm]{geometry}
\usepackage[letterpaper,margin=1in]{geometry} % e. Set margins of 1 inch (2.54 cm.) on all four sides of the paper. 
\usepackage{mathptmx} % d. ...in a simple roman face except where indicated below (§3). 
\usepackage[singlespacing]{setspace} % Set line spacing to 1 throughout the document
\usepackage{fancyhdr} 

% Set the headheight to at least 14.49998pt
\setlength{\headheight}{14.49998pt}

% Optionally adjust \topmargin if necessary
% \addtolength{\topmargin}{-2.49998pt}
\usepackage{relsize}

\usepackage[bottom]{footmisc}
\usepackage{tabularx}
    
\pagestyle{empty}        % No page numbers

% Set paragraph indentation to 1.5cm
\setlength{\parindent}{2em}

%%%Using XeTeX (xelatex, lulatex):
%\usepackage{polyglossia}
%\usepackage{fontspec}
%\usepackage{xunicode}
%\usepackage{xltxtra}
\usepackage{url}
\usepackage{hyperref}
\usepackage[english,german]{babel}

\usepackage{graphicx}

%\setmainfont[Mapping=tex-text]{Linux Libertine O} %Falls nicht vorhanden müssen die LinLibertine-ttf-Dateien nach C:\windows\fonts verschoben werden

\usepackage{booktabs}    % For nice-looking tables
\usepackage{expex}

\usepackage{acronym}
\usepackage{multicol}

\author{Ruben Triwari}

\usepackage{scrhack} % Recommended to avoid potential conflicts
\usepackage{microtype}


%\usepackage[acronym,xindy,toc]{glossaries} % TODO: include "acronym" if glossary and acronym should be separated
%\makeglossaries
%\loadglsentries{pages/glossary.tex} % important update for glossaries, before document

\usepackage{ragged2e}

% Set up hyphenation rules for the language package when mistakes happen
\babelhyphenation[english]{
an-oth-er
ex-am-ple
}


\begin{document}

\selectlanguage{german}


\begin{center}\uppercase{Ludwig-Maximilians-Universität München}\end{center}
\begin{center}
  \uppercase{Lehr- und Forschungseinheit für 
      Theoretische Informatik und Theorembeweisen}
\end{center}

\vspace*{10mm}
\begin{center}
\includegraphics[height=40mm]{sigillum.png}
\end{center}
\vspace*{10mm}

\title{Titel der Arbeit}
\date{\vspace{-5ex}}
{\let\newpage\relax\maketitle}
\thispagestyle{empty}
\begin{center}
\begin{large}
\begin{Large}
Thesis type (Bachelor's Thesis / Master's Thesis, \ldots)\\
\end{Large}
im Studiengang 'Studiengang (Informatik, Informatik plus Mathematik, \ldots)' \\
\end{large}
\end{center}
\vspace{1cm}
\begin{center}
\begin{large}
Betreuer: Prof. Jasmin Blanchette\\
\end{large}
\end{center}
\begin{center}
\begin{large}
Mentor: Name des Mentors/der Mentorin\\
\end{large}
\end{center}


\begin{center}
\begin{large}
Ablieferungstermin: \date{\today} \\
\end{large}
\end{center}

\vspace{1,5cm}
\cleardoublepage{}

\thispagestyle{empty}
\vspace*{0.8\textheight}
\noindent
\makeatletter
\begin{center}
\iflanguage{english}{
{\normalfont\bfseries} Disclaimer
}
{}
\iflanguage{german}{
{\normalfont\bfseries} Erklärung
}
{}
\end{center}
\begin{flushleft}
\iflanguage{english}{
{I confirm that this \MakeLowercase{Thesis type} is my own work and I have documented all sources and material used.}
}
{}
\iflanguage{german}{
{Ich versichere, dass ich diese Arbeit selbstständig verfasst und nur die angegebenen Quellen und Hilfsmittel verwendet habe.}
}
{}
\makeatother

\vspace{15mm}
\noindent
Munich, \date{\today} \hspace{50mm} Author
\end{flushleft}
\cleardoublepage{}

\thispagestyle{empty}

\vspace*{20mm}

\begin{center}
\makeatletter

\iflanguage{english}{
{\normalfont\bfseries} Acknowledgments
}
{}
\iflanguage{german}{
{\normalfont\bfseries} Danksagungen
}
{}
\makeatother
\end{center}

\vspace{10mm}

%TODO: Acknowledgments

\cleardoublepage{}
 % TODO: if you don't have anyone to thank for or don't wish to publish it, comment this line out.
% TODO: decide if english or german
\iflanguage{english}{
\section*{Abstract}
}
{}
\iflanguage{german}{
\section*{Kurzfassung}
}
{}
%TODO: Abstract
Artificial Intelligence (AI) has emerged as a transformative force in the field of healthcare, offering unprecedented opportunities to enhance patient care, streamline processes, and improve outcomes. This research paper explores the profound impact of AI on healthcare delivery by examining its applications in various facets of the healthcare ecosystem, from diagnostic accuracy and treatment optimization to administrative efficiency and patient engagement. We review recent case studies and data-driven insights to illustrate the tangible benefits and challenges associated with the integration of AI technologies in healthcare settings. Additionally, we discuss ethical considerations and the need for regulatory frameworks to ensure the responsible and equitable deployment of AI in healthcare. This study highlights the potential of AI to revolutionize healthcare delivery and underscores the importance of a thoughtful and ethical approach to its implementation.

\textbf{Keywords}: Artificial Intelligence, Healthcare Delivery, Patient Care, Diagnostic Accuracy, Treatment Optimization, Ethical Considerations, Regulatory Frameworks.


\newpage
\tableofcontents
\newpage

\setcounter{page}{1}
\pagestyle{fancy}
\fancyhf{}
\fancyhead[R]{\thepage}
\renewcommand{\headrulewidth}{0pt} %obere Trennlinie

\section{Introduction}
This document serves as a template for bachelor or master theses in the 'Chair of Theoretical Computer Science and Theorem Proving' at \gls{lmu}. 
By commenting one of the commands $selectlanguage\{english\}$ or $selectlanguage\{german\}$ the documents language is changed.
\\The template follows the guidelines of the journal \emph{language} (\url{http://www.linguisticsociety.org/sites/default/files/style-sheet.pdf}).

\section{Notes on form}
\subsection{Formatting} 
This LaTeX template uses the following formatting:
\begin{itemize}
\setlength{\itemsep}{0pt}
	\item font: Linux Libertine O (alternatively: Times New Roman)
	\item font size: 12 pt
	\item left and right margin: 3.5 cm, top and bottom margin: 3 cm
    \item align: left
	\item line spacing: one and a half (alternative: 15 pt line spacing with 12 pt font size)
\end{itemize}

\noindent
When implementing the specifications in Word, it is essential to define style sheets.

\subsection{Citation} 
\label{sec:cit}

The citation method follows the author-year system. Place reference is in the text, footnotes should only be used for explanations and comments. The following notes are taken from the \emph{language} bibliography template from \url{ron.artstein.org}:\newline

\noindent
The \emph{Language} style sheet makes a distinction between two kinds of in-text citations: citing a work and citing an author.
\begin{itemize}
\item Citing a work:
  \begin{itemize}
    \setlength{\itemsep}{0pt}
    \setlength{\parsep}{0pt}
  \item Two authors are joined by an ampersand (\&).
  \item More than two authors are abbreviated with \emph{et al.}
  \item No parentheses are placed around the year (though parentheses
    may contain the whole citation). 
  \end{itemize}
\item Citing an author:
  \begin{itemize}
    \setlength{\itemsep}{0pt}
    \setlength{\parsep}{0pt}
  \item Two authors are joined by \emph{and}.
  \item More than two authors are abbreviated with \emph{and colleagues}.
  \item The year is surrounded by parentheses (with page numbers, if
    present).
  \end{itemize} 
\end{itemize}
To provide for both kinds of citations, \verb+language.bst+ capitalizes on the fact that \verb+natbib+ citation commands come in
two flavors. In a typical style compatible with \verb+natbib+, ordinary commands such as \verb+\citet+ and \verb+\citep+ produce short
citations abbreviated with \emph{et al.}, whereas starred commands such as \verb+\citet*+ and \verb+\citep*+ produce a citation with a
full author list. Since \emph{Language} does not require citations with full authors, the style \verb+language.bst+ repurposes the starred commands to be used for citing the author. The following table shows how the \verb+natbib+ citation commands work with \verb+language.bst+.
\begin{center}
  \begin{tabular}{lll}
    \toprule
    Command & Two authors & More than two authors \\
    \midrule
    \verb+\citet+ & \citet{hale} & \citet{sprouse} \\
    \verb+\citet*+ & \citet*{hale} & \citet*{sprouse} \\
    \addlinespace
    \verb+\citep+ & \citep{hale} & \citep{sprouse} \\
    \verb+\citep*+ & \citep*{hale} & \citep*{sprouse} \\
    \addlinespace
    \verb+\citealt+ & \citealt{hale} & \citealt{sprouse} \\
    \verb+\citealt*+ & \citealt*{hale} & \citealt*{sprouse} \\
    \addlinespace
    \verb+\citealp+ & \citealp{hale} & \citealp{sprouse} \\
    \verb+\citealp*+ & \citealp*{hale} & \citealp*{sprouse} \\
    \addlinespace
    \verb+\citeauthor+ & \citeauthor{hale} & \citeauthor{sprouse} \\
    \verb+\citeauthor*+ & \citeauthor*{hale} & \citeauthor*{sprouse} \\
    \verb+\citefullauthor+ & \citefullauthor{hale} & \citefullauthor{sprouse} \\
    \bottomrule
  \end{tabular}
\end{center}
Authors of \emph{Language} articles would typically use \verb+\citet*+, \verb+\citep+, \verb+\citealt+ and \verb+\citeauthor*+, though they
could use any of the above commands. There is no command for giving a full list of authors.

\subsection{Bibliography}
The bibliography of this template includes the references of the \emph{language} stylesheet as a sample bibliography.

\pagebreak

\section{General Addenda}

If there are several additions you want to add, but they do not fit into the thesis itself, they belong here.

\subsection{Detailed Addition}

Even sections are possible, but usually only used for several elements in, e.g.\ tables, images, etc.

\section{Figures}
\subsection{Example 1}
\subsection{Example 2}


\pagebreak

\microtypesetup{protrusion=false}
\listoffigures{}
\listoftables{}
\microtypesetup{protrusion=true}

\clearpage
\printglossaries

\pagebreak

\addcontentsline{toc}{section}{Literatur}
\pagestyle{fancy}

\bibliographystyle{language-dt} %using language.bst
\bibliography{bibliography} %bib-filename

\nocite{*} %List all bib-entries

\end{document}
